\section{abstract}

Das Dicke-Modell beschreibt ein gekoppeltes System aus einem Spin beliebiger Länge und einer bosonischen Anregung.
Ursprünglich zur quantenmechanischen Beschreibung der Strahlung eines Gases entwickelt, ist es unter Anderem durch den Quantenphasenübergang und die Möglichkeit zur analytischen Behandlung des klassischen Grenzfalls von Interesse. 

Ich befasse mich mit einem verallgemeinerten Dicke-Modell, in dem die Paritätssymmetrie, die den Phasenübergang verursacht, gebrochen werden kann.
Die Symmetrie zu brechen ermöglicht es, den Phasenübergang genauer zu untersuchen und vorangegangene Forschung zu verallgemeinern, beispielsweise das Untersuchen von Quantenchaos mittels Quanteninvarianten.

Um die quantenmechanischen Eigenschaften mit dem klassischen Grenzfall zu vergleichen, wird eine Poincar\'{e}-Husimi-Projektion und die Zeitentwicklung von kohärenten Produktzuständen untersucht.

Zur Diagonalisierung der dünnbesetzten Matrizen wird der FEAST-Algorithmus verwendet, die Zeitentwicklung wird mittels Chebyshev-Entwicklung realisiert.\\
\ \\
\ \\
\ \\
The Dicke model describes a coupled System of a spin of arbitrary length and a bosonic mode.
Originally intended to describe  a radiating gas, it became of interest for its quantum phase transition and analytical approachability in the classical limit. 

I consider a generalized model, in which the parity symmetry, which causes the phase transition, can be broken.
Breaking the parity allows us to study the nature of the phase transition and to generalize previous research, for example analyzing quantum chaos via quantum invariants.

For the transition to the classical limit, a Poincar\'{e} Husimi projection is used and the time evolution of initially coherent states is analyzed.

Numerical methods include sparse matrix diagonalization using the FEAST algorithm and time evolution via Chebyshev expansion.
