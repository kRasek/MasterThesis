\section{Klassischer Limes}
Zunächst soll ein klassisches System bestimmt werden, das dem quantenmechanischem Modell im Limes $j\rightarrow \infty, \hbar \rightarrow 0$ entspricht.
Dies ist ein System bestehend aus einem Drehimpuls der Länge $s = \hbar\sqrt{j(j+1)} \approx \hbar j$, die im Grenzwert konstant bleiben soll, und einem Oszillator mit Orts- und Impulskoordinaten $Q$ und $P$.
\subsection{Hamiltonfunktion}
Zur Bestimmung einer klassischen Hamiltonfunktion werden die üblichen Ersetzungen angewandt
\begin{align*}
  J_{\pm} &= J_x \pm i J_y~~~~~~~~~~~ a = \frac{1}{\sqrt{2}} (Q+iP)~~~~~~~ a^\dagger = \frac{1}{\sqrt{2}} (Q-iP)~~.\\
\end{align*}
Um dies auf dimensionsbehaftete Größen zurückzuführen, werden folgende Ersetzungen gemacht:
\begin{align*}
  L_i &= \hbar J_i,~~~~~  q=x_0 Q,~~~~~p= \frac{\hbar}{x_0} P\\
  \hbar \omega a^\dagger a &=\hbar \omega \frac{1}{2} (P^2+Q^2-1) = \frac{1}{2} m \omega^2 q^2 + \frac{1}{2m}p^2 -\cancelto{\text{0 im Limes}}{\hbar\omega/2}\\
\end{align*}
mit $x_0  =  \sqrt{\frac{\hbar}{m\omega}}$.
 Da $x_0$ für $\hbar\rightarrow 0$ verschwindet, verwende \linebreak  $\tilde{x} = \sqrt{\frac{s}{m\omega}} = \sqrt{j} x_0$ welches im Limes konstant bleibt.
Damit ist
\begin{align*}
  \frac{\hbar}{\sqrt{j}} a J_+ &= \frac{1}{\sqrt{2}}(\frac{q}{\tilde{x}} + i \frac{\tilde{x}}{s} p) (L_x + i L_y)\\
  \frac{\hbar}{\sqrt{j}} a J_- &=\frac{1}{\sqrt{2}}(\frac{q}{\tilde{x}} + i \frac{\tilde{x}}{s} p)(L_x - i L_y)\\
  \frac{\hbar}{\sqrt{j}} a^\dagger J_- &=\frac{1}{\sqrt{2}}(\frac{q}{\tilde{x}} - i \frac{\tilde{x}}{s} p)(L_x - i L_y)\\
  \frac{\hbar}{\sqrt{j}} a^\dagger J_+ &=\frac{1}{\sqrt{2}}(\frac{q}{\tilde{x}} - i \frac{\tilde{x}}{s} p)(L_x + i L_y)~~.\\
\end{align*}
Im klassischen Limes $(\langle ab\rangle = \langle a\rangle \langle b\rangle)$ folgt die Hamiltonfunktion
\begin{align*}
 H(\vec{L},p,q) =& \omega_0 (\cos \alpha L_z + \sin\alpha L_x)+ \frac{1}{2} m \omega^2 q^2 + \frac{1}{2m}p^2 \\
&+ \sqrt{2}g\left( \cos(\delta) ( L_x \frac{q}{\tilde{x}} - L_y \frac{\tilde{x}}{s} p) 
+ \sin(\delta)( L_x \frac{q}{\tilde{x}} + L_y \frac{\tilde{x}}{s} p)\right)
\end{align*}
Zurück zu klassischen dimensionslosen Größen: $l_i = \frac{L_i}{s}$, $Q=\frac{q}{\tilde{x}}$, $P = \frac{\tilde{x}}{s}p$, $h = H/(s\omega_0)$. Beachte hierbei einen Skalierungsfaktor $\frac{1}{\sqrt{j}}$ gegenüber den dimensionslosen Operatoren die durch $a$ und $a^\dagger$ definiert sind.
\begin{align}
 h(\vec{l},P,Q) &= \cos\alpha l_z + \sin \alpha l_x+ \frac{1}{2} \underbrace{\frac{\omega}{\omega_0}}_{\nu} (P^2 + Q^2) 
  + \underbrace{\sqrt{2} \frac{g}{\omega_0}}_{g_0} \left( \cos\delta (l_x Q - l_y P) + \sin\delta (l_x Q+l_y P)\right) \nonumber\\ 
  &= \cos\alpha l_z + \sin \alpha l_x + \frac12 \nu (P^2+Q^2) + g_0\left((\sin\delta + \cos\delta)l_x Q + (\sin\delta -\cos\delta)l_y P \right) \label{Hamiltonian}
\end{align}
In der üblichen Notation ist
\begin{align}
  g_0 = \gamma\sqrt{\frac{\nu}{2}}~~,
\end{align}
sodass im Standard-Dicke-Modell ($\delta = \pi/4,\alpha = 0$) der Phasenübergang bei $\gamma = 1$ auftritt.  
\subsection{Energieminimum}
Energieminimum mit Nebenbedingung $l_x^2 + l_y^2+l_z^2 = 1$. Minimiere also 
\begin{equation*}
F(\vec{l}, P,Q, \lambda) = h(\vec{l}, P,Q)+\lambda (l_x^2 + l_y^2+l_z^2 - 1) ~~.
\end{equation*}
Aus $\nabla F = 0$ ergeben sich die Gleichungen:
\begin{align}
  0 &= \nu P + g_0 l_y (\sin \delta - \cos \delta)\label{P}\\
  0 &= \nu Q + g_0 l_x (\sin \delta + \cos \delta)\label{Q}\\
  0 &= g_0 Q (\sin \delta + \cos\delta) + 2\lambda l_x + \sin\alpha\label{lx}\\
  0 &= g_0 P (\sin \delta - \cos\delta) + 2\lambda l_y\label{ly}\\
  0 &= \cos\alpha + 2\lambda l_z\label{lz}\\
  0 &= l_x^2 + l_y^2 + l_z^2 -1 \label{lambda}
\end{align}
Multipliziere \eqref{P} mit $P$ und \eqref{ly} mit $l_y$ und eliminiere die Sinus- und Cosinus-Anteile um auf die Gleichungen
\begin{align}
 2\lambda l_y^2 = \nu P^2 \label{P^2}\\
2\lambda l_x^2 + \sin\alpha l_x  = \nu Q^2 \label{Q^2}
\end{align}
zu kommen. Die zweite folgt analog aus den Gleichungen \eqref{Q} und \eqref{lx}. \\


\subsubsection{ Mit Parität: $\sin\alpha = 0$}
Zunächst für den Fall mit erhaltener Parität:\\
\eqref{P} nach $P$ umformen in \eqref{ly}, und \eqref{Q} nach $Q$ umformen in \eqref{lx} liefern
\begin{align*}
  2\lambda l_y = \frac{g_0^2}{\nu}(\sin\delta - \cos\delta)^2 l_y\\
  2\lambda l_x = \frac{g_0^2}{\nu}(\sin\delta + \cos\delta)^2 l_x~~.
\end{align*}
Damit beides erfüllt werden kann, muss $l_x$ oder $l_y$ (oder beide) $0$ sein.

Fallunterscheidung:
\begin{itemize}
 \item $l_x=l_y=0$:\\
 damit $l_z=\pm1$ , $P=Q=0$,  $\lambda = \mp \frac{1}{2}$, $h=\pm 1$
\item $l_x=0, \lambda = \frac{g_0^2}{2\nu}(\sin\delta - \cos\delta)^2 =:\lambda_- = \frac{\gamma^2}{4}(\sin\delta - \cos\delta)^2$:\\
Aus \eqref{lz} folgt $l_z = -\frac{1}{2\lambda}$, damit $l_y = \pm \sqrt{1-(\frac{1}{2\lambda})^2}$\\
Dann folgt aus \eqref{P} und \eqref{Q} $Q=0$ und $P=\mp \frac{g_0}{\nu}(\sin\delta - \cos\delta) \sqrt{1-(\frac{1}{2\lambda})^2}$.
Die Energie ist dann
\begin{align*}
  h &= -\frac{1}{2\lambda}  + \frac{1}{2} \nu \frac{2\lambda}{\nu}\left(1-(\frac{1}{2\lambda})^2\right) 
- \overbrace{\frac{g_0^2}{\nu} (\sin\delta-\cos\delta)^2}^{ = 2\lambda}\left(1-(\frac{1}{2\lambda})^2\right)\\ 
  &= -\frac{1}{2\lambda} + \lambda -\frac{1}{4\lambda} - (2\lambda -\frac{1}{2\lambda})=-\lambda -\frac{1}{4\lambda}
\end{align*}
\item $l_y=0, \lambda = \frac{g_0^2}{2\nu}(\sin\delta + \cos\delta)^2 =:\lambda_+$:\\
Aus \eqref{lz} folgt $l_z = -\frac{1}{2\lambda}$, damit $l_x = \pm \sqrt{1-(\frac{1}{2\lambda})^2}$
Dann folgt aus \eqref{P} und \eqref{Q} $P=0$ und $Q=\mp \frac{g_0}{\nu}(\sin\delta + \cos\delta)\sqrt{1-(\frac{1}{2\lambda})^2}$.
Die Energie ist dann
\begin{align*}
  h &= -\frac{1}{2\lambda}  + \frac{1}{2} \nu \frac{2\lambda}{\nu}\left(1-(\frac{1}{2\lambda})^2\right) 
- \overbrace{\frac{g_0^2}{\nu} (\sin\delta+\cos\delta)^2 }^{2\lambda}\left(1-(\frac{1}{2\lambda})^2\right)\\ 
  &= -\frac{1}{2\lambda} + \lambda -\frac{1}{4\lambda} - (2\lambda -\frac{1}{2\lambda})=-\lambda -\frac{1}{4\lambda}
\end{align*}
\end{itemize}
Für $\lambda \geq 0$ sind die letzteren Energien kleiner als $-1$. 
Beachte auch die Einschränkung $|\lambda|\geq 1/2$ (wegen $|l_z|\leq 1$) und $\lambda > 0$ (da $\nu$ positiv).\\
Das Energieminimum wird also erreicht entweder bei \\
$P=Q=l_x=l_y=0, l_z = -1 = h$, bei 
\begin{align*}
  l_z & =-\frac{1}{2\lambda_+},~~~~~l_{x} &= \pm \sqrt{1-(\frac{1}{2\lambda_+})^2},~~~~~l_{y} &= 0 ,~~~~~ P&=0,~~~~~Q=\mp \frac{g_0}{\nu}(\sin\delta + \cos\delta) \sqrt{1-(\frac{1}{2\lambda_+})^2} 
\end{align*}
oder bei
\begin{align*}
  l_z & =-\frac{1}{2\lambda_-},~~~~~l_{y} &= \pm \sqrt{1-(\frac{1}{2\lambda_-})^2},~~~~~l_{x} &= 0, ~~~~~ Q&=0,~~~~~P= \mp \frac{g_0}{\nu}(\sin\delta - \cos\delta) \sqrt{1-(\frac{1}{2\lambda_-})^2} 
\end{align*}
Wenn $\lambda_\pm>\frac12$ wird das absolute Minimum in einem nichttrivialen Fall erreicht.
Da $-\lambda -\frac{1}{4\lambda}$ für $\lambda>\frac{1}{2}$ streng monoton fallend ist, wird der Fall mit dem größeren $\lambda$ die geringere Energie aufweisen.
Es gilt 
\begin{equation}
 \lambda_+ > \lambda_- \Leftrightarrow \sin(2\delta)>0
\end{equation}


Für das Standard-Dicke-Modell ($\delta = \pi/4$) gilt
\begin{equation}
 \lambda_+>\frac{1}{2} \Leftrightarrow \gamma>1~,
\end{equation}
der Phasenübergang tritt also bei $\gamma_c=1$ auf.
Im Allgemeinen ist Fall ist die Bedingnung
\begin{equation}
 \lambda_\pm > \frac{1}{2} \Leftrightarrow \gamma>\frac{\sqrt{2}}{|\sin\delta \pm\cos\delta|},
\end{equation}


\subsubsection{ Ohne Parität: $\sin\alpha \neq 0$}
Es bleiben die Gleichungen:
\begin{align}
  2\lambda l_y^2 &= \nu P^2 \nonumber\\
  2\lambda l_x^2 + \sin\alpha l_x  &= \nu Q^2\nonumber\\ 
  2\lambda l_x + \sin\alpha&= \frac{g_0^2}{\nu}(\sin\delta + \cos\delta)^2 l_x\label{Gleichung fuer lx}\\
  2\lambda l_y &= \frac{g_0^2}{\nu}(\sin\delta - \cos\delta)^2 l_y\nonumber\\
\end{align}
Die letzte liefert wieder die Fallunterscheidung $l_y =$ oder $\neq 0$. Außerdem gilt weiterhin $|\vec{l}|=1, l_z = -\cos\alpha/(2\lambda)$.
\\ 
\\

Fall $l_y=0$:\\
Dann ist auch $P=0$. $l_x = \pm \sqrt{1-\frac{\cos^2\alpha}{4\lambda^2}}$. $\lambda$ bestimmt sich dann aus der dritten Gleichung:
\begin{equation*}
  2\lambda \sqrt{1-\frac{\cos^2\alpha}{4\lambda^2}} \mp \sin\alpha = \frac{g_0^2}{\nu} \sqrt{1-\frac{\cos^2\alpha}{4\lambda^2}} (\sin\delta + \cos\delta)^2=: 2\sqrt{1-\frac{\cos^2\alpha}{4\lambda^2}}~~\lambda_+
\end{equation*}
Das lässt sich umformen zu einem Polynom vierten Grades in $\lambda$
\begin{equation*}
  \lambda^4-2\lambda_+ \lambda^3 +(\lambda_+^2-\frac14)\lambda^2 + \frac{\cos^2\alpha}{2}\lambda_+ \lambda -\frac{\cos^2\alpha}{4}\lambda_+^2=0
\end{equation*}
Beachte, dass sich für $\sin\alpha = 0$, die Fallunterscheidung $\lambda=\pm1/2$ oder $\lambda = \lambda_+$ ergibt, was zu den gleichen Ergebnissen wie oben führt.
\\

Fall $l_y \neq 0$:
\begin{align*}
  \lambda &= \frac {g_0^2}{2\nu}(\sin\delta -\cos\delta)^2 = \frac {g_0^2}{2\nu}(1-\sin(2\delta))=:\lambda_0(1-\sin(2\delta)) = \lambda_-\\
  l_x &= -\sin\alpha \left( 2\lambda - \frac{g_0^2}{\nu}(\sin\delta + \cos\delta)^2 \right)^{-1} = -\sin\alpha \frac{\nu}{g_0^2} \left( (\sin\delta - \cos\delta)^2 - (\sin\delta + \cos\delta)^2\right)^{-1}\\
  &= \sin\alpha \frac{\nu}{g_0^2} \frac{1}{4\sin\delta\cos\delta} = \sin\alpha \frac{\nu}{ 2 g_0^2} \frac{1}{\sin(2\delta)} =  \frac{\sin\alpha}{4} \frac{1}{\lambda_0 -\lambda}\\
  l_y &= \pm \sqrt{1 - \frac{\cos^2\alpha}{4\lambda^2} - \frac{\sin^2\alpha}{16} \frac{1}{(\lambda_0-\lambda)^2}}\\
\end{align*}
Verwende die Gleichungen \eqref{P} bis \eqref{lambda} um die Minimalenergie abhängig von $\lambda$ auszurechnen. Ersetze mit \eqref{P} und \eqref{Q} die $\delta$-abhängigen Terme in \eqref{Hamiltonian}:
\begin{align*}
 h_{min}  = \cos\alpha l_z + \sin\alpha l_x - \frac{\nu}{2}(P^2+Q^2)
\end{align*}
Weiter mit \eqref{P^2} und \eqref{Q^2} 
\begin{align*}
  h_{min} &= -\frac{\cos^2\alpha}{2\lambda} + \sin\alpha l_x - (\underbrace{\lambda l_x^2 + \lambda l_y^2}_{\lambda(1-l_z^2)} + \frac{\sin\alpha}{2} l_x)\\
  &= - \frac{\cos^2\alpha}{4\lambda} + \frac{\sin\alpha}{2} l_x - \lambda 
\end{align*}
und mit Gleichung \eqref{Gleichung fuer lx} ergibt sich
\begin{align}
  h_{min} = -\frac{\cos^2\alpha}{4\lambda} + \frac{\sin^2\alpha}{4(\lambda_+ - \lambda)} - \lambda~~.
\end{align}
\subsection{Bewegungsgleichungen}
$P,Q$ aus hamiltonschen Bewegungsgleichungen:
\begin{align*}
  \dot{P} &= - \frac{\partial h}{\partial Q} = - \nu Q - g_0 l_x (\sin\delta + \cos\delta)\\
  \dot{Q} &= + \frac{\partial h}{\partial P} = + \nu P + g_0 l_y (\sin\delta - \cos\delta) 
\end{align*}
Da die $l_i$ keine kanonischen Variablen sind, müssen die DGLs z.B. aus dem Ehrenfest-Theorem hergeleitet werden, mit Hamiltonian \eqref{Hamiltonian}, wenn man sich alles wieder als Operatoren denkt:
\begin{equation*}
  \frac{d}{dt}\langle l_j \rangle = i \langle[h,l_j]\rangle + \cancelto{0}{\langle\frac{\partial l_j}{\partial t}\rangle}
\end{equation*}
Damit ergibt sich
\begin{align*}
  \dot{l_x} &= + g_0 (\sin\delta - \cos\delta) P l_z - l_y\\
  \dot{l_y} &= - g_0 (\sin\delta + \cos\delta) Q l_z  + l_x -\sin\alpha l_z\\
  \dot{l_z} &= + g_0 \left[ (\sin\delta + \cos\delta) Q l_y - (\sin\delta - \cos\delta)P l_x\right] + \sin\alpha l_y~~.
\end{align*}
