\section{Basics}
The Standard Dicke hamiltonian reads
\begin{equation}
 H=\hbar \left[ \omega_0 \cos\alpha J_z + \omega a^\dagger a + \frac{g}{\sqrt{j}} J_x (a^\dagger + a) \right]~~.
\end{equation}
It describes a single bosonic mode of frequency $\omega$ with creation and annihilation operators $a^\dagger$ and $a$, and a spin length $j$.
The spin operators $ J_i$ follow the standard spin algebra $[J_k,J_l] = i \sum\limits_m \epsilon_{klm} J_m$. 
For small $j$, these operators may describe a single atom or molecule, for larger $j$ it can describe an ensemble of $N$ spin $\frac{1}{2}$ paricles of which only the subspace of maximum total spin $j = \frac{N}{2}$ is concidered.
This assumption is justified, because the ground state of the Spin ensemble, disregarding the coupling, is in this subspace and no operators that change the total spinare present.


By rewriting the Spin operators $J_x$ and $J_y$ in spin rising and lowering operators $J_\pm = J_x \pm i J_y$, one can identify resonant and antiresonant terms in the coupling.
The resonant terms $a J_+$ and $a^\dagger J_-$ conserve the number of excitations $N_\text{exc} = a^\dagger a + J_z + j$ in the system.
The antiresonant terms $a J_-$ and $a^\dagger J_+$ do not. 
Thereby it is an often used and justified method to neglect the antiresonant terms of the coupling - the rotating wave approximation [?].
Not only does the existence of a second constant of motion beside the energy have significant influence on the dynamics of the system, but the breakdown of the Hilbertspace into infinitely many, independent subspaces of dimension $2j+1$, makes the numerical handling of the problem a lot easier.
Nonetheless there are many situations in which the rotating wave approximation is not applicable[?].
To study the influence of this approximation we will consider a generalized form of the Standard Dicke Hamiltonian by using the angle $\delta$ to set the ratio of resonant and antiresonant terms.
The coupling term then reads $\frac{g}{\sqrt{j}}\left( \cos(\delta) (aJ_+ + a^\dagger J_-) +\sin(\delta) (a^\dagger J_+ +a J_-) \right)$, which can shift from rotating wave approximation $\delta = 0$ to the Standard coupling $\delta = \pi/4$ to a purely antiresonant coupling $\delta = \pi/2$.
Neglecting the rotating wave approximation, what remains is the parity symmetry $P = \exp(i\pi N_\text{exc})$, which can be interpreted as whether the number of excitations is even or odd.
One can see, that the antiresonant terms only ever create or annihilate two excitations, so the parity remains conserved.
Whether this has a significant influence on the dynamics of the system is subject of this thesis, but one way or the other the Hilbert space separates into two independent subspaces.
Their dimension is infinite but after applying a suitable truncation of the bosonic Hilbertspace, exploiting the parity can lead to significantly less numeric efforts.
A second geralization is now applied in order to break the parity.
For this a term that creates or annihilates one excitation is needed.
We use a second angle $\alpha$ to shift the constant magnetic field, that splits the energy levels of the Spin from the $z$ axis into the $x-z$ plain, so that the generalized Dicke Hamiltonian considered in this thesis reads
\begin{equation}
 H=\hbar \left[ \omega_0 (\cos\alpha J_z + \sin \alpha J_x )+\omega a^\dagger a + \frac{g}{\sqrt{j}}\left( \cos(\delta) (aJ_+ + a^\dagger J_-) +\sin(\delta) (a^\dagger J_+ +a J_-) \right) \right]~~. \label{eq:H}
\end{equation}

% Different Limits
One of the important qualties of the Dicke model is, that it has a well defined classical limit[?].
There are different limits to be considered. 
The thermodynamic limit describes the Dicke model in the background of $N$ particles in, e.g. an optical cavity, the limit being $N\rightarrow \infty$ as is the usual thermodynamic limit in solid state or statistical physics[?].
It is to be noted, that the system remains quantum mechanical.
The classical limit yields very similar results, but applies a limit, where the Spin length becomes infinite, resulting in infinitely many energy levels and possible orientations of the spin -as a classical angular momentum- , while $\hbar$ is sent to zero in a way, that $\hbar j = s$ remains constant.
This limit is usually referred to as the classical limit, sometimes called the classical spin limit, because it is also possible to apply a limit, that makes the oscillator classical, the classical oscillator limit [? Fehske et al und noch jemand, vielleicht?].
Here the frequency, i.e. the energy spacing, of the oscillator is sent to zero, so that the oscillator can be described by a classical continuous harmonic oscillator, while the Spin remains quantum mechanical.
This limit will only briefly be covered in this thesis (if at all, as of 26.06.), because the classical spin limit yields enough classical and, more importantly, analytically approachable results.
It is worth noting, that even in the classical spin limit, henceforth classical limit, the bosonic part of the Hamiltonian becomes a classical oscillator, since the energy spacing vanishes as $\hbar$ approaches zero.

%special case \alpha = pi/2
A simple special case is the combination of standard Dicke coupling, i.e. $\delta = \pi/4$, and $\alpha = \pi/2$. That way the hamiltonian only includes the $x$ component of the spin and is exactly solvable for any spin length and parameter choice.
That way the spin operator $J_x$ commutes with the hamiltonian and the spin part of any energy eigenstate is an eingenstate of $J_x$. 
The oscillator part then is only a shifted oscillator, where the offset depends of the spin quantum number.


