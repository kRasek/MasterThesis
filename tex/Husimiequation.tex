\section{Husimi equation}
To compare quantum mechanical states with classical trajectories or phase space points, the Husimi projection 
\begin{equation}
  Q(\alpha, z) = \frac{2j + 1}{\pi^2 \left(1 + |z|^2\right)^2} \braket{\alpha, z|\rho|\alpha,z}
\end{equation}
is a suitable way.
Here $\rho$ is the density matrix, in case of a pure state it is $\rho = \ket{\Psi}\bra{\Psi}$.
This projection basically projects onto a coherent product state, thus giving the closest possible analogon to a classical propability density.
The variables $\alpha$ and $z$ are each complex variables, and thus represent two phase space coordinates each.
Thus the Husimi projection is fourdimensional and unsuitable to directly represent the quantum mechanical state (in picture), but it can still be studied regarding fundamental properties.
Especially the time evolution of the Husimi projection yields notable features.
Starting with the von Neumann equation for the time evolution of an arbitrary density matrix a differential equation for the time evolution of the Husimi projection can be derived.
Straightforward insertion of the von Neumann equation $\dot{rho} = -\frac{i}{\hbar} \left[H,\rho\right]$ into the definition of the Husimi projection gives 
\begin{equation}
  \dot{Q} (\alpha, z) = -\frac{i}{\hbar} \frac{2j+1}{\pi^2 \left(1 + |z|^2\right)^2} \braket{\alpha, z|\left[H,\rho\right]|\alpha,z}~~.
\end{equation}
This can be rewritten as
\begin{equation}
  \dot Q(\alpha, z) = \frac{i (2j + 1)}{\hbar \pi^2 \left(1 + |z|^2\right)^2} \Tr\left(\rho \left[H,\ket{\alpha,z}\bra{\alpha,z}\right] \right)~~.\label{eq:Qdot}
\end{equation}
Furthermore we can avoid the commutator by using $\Tr\left(\rho \left[H, \ket{\alpha,z}\bra{\alpha,z} \right]\right)  = \Tr \left( \rho H \ket{\alpha,z}\bra{\alpha,z} \right) - c.c.$.

Take as an initial state a product coherent state, so that $\rho = \ket{\alpha', z'}\bra{\alpha', z'}$.
In order to calculate a differential eqation for $Q(\alpha, z)$ one must now calculate the trace in equation \eqref{eq:Qdot} and express occuring operators as derivations concerning $\alpha$ and $z$.
These relations like
\begin{equation}
  a^\dagger \ket{\alpha}\bra{\alpha} = \left(\partial_\alpha + \alpha^*\right)\ket{\alpha}\bra{\alpha}
\end{equation}
and
\begin{equation}
 J_+ \ket{z}\bra{z} = \left( -z^2 \partial_z + 2j\frac{z}{1+|z|^2}\right)\ket z \bra z
\end{equation}
can be calculated straightforward\textcolor{red}{expamples of which might be given in the appendix}.
Deviations concerning complex variables shall be defined as
\begin{equation}
  \frac{\partial f}{\partial \alpha} = \frac{1}{2}\left( \frac{\partial f}{\alpha_r} - i \frac{\partial f}{\alpha i}\right)
\end{equation}
and 
\begin{equation}
  \frac{\partial f}{\partial \alpha^*} = \frac{1}{2}\left( \frac{\partial f}{\alpha_r} + i \frac{\partial f}{\alpha i}\right)
\end{equation}
where $\alpha_r$ is the real and $\alpha_i$ the imaginary part of $\alpha$.
Thereby $\partial_\alpha \alpha =\ partial_\alpha^* \alpha^* = 1$ and $\partial_\alpha^* \alpha = \partial_\alpha \alpha^* = 0$, thus a complex variable and its complex conjugate can be considered independent variables concerning deviations.
With this the general strategy in calculating the differential equation is to substitute operators in the Hamiltonian by derivations, which are independent of the trace, and identify 
\begin{equation}
  \frac{2j + 1}{\pi^2 \left(1 + |z|^2\right)^2} \Tr \left[ \rho \ket{\alpha,z}\bra{\alpha,z} \right] = Q(\alpha, z)
\end{equation}
which uses $\ket{\alpha,z}\bra{\alpha,z} = \ket{\alpha,z}\bra{\alpha,z}\ket{\alpha,z}\bra{\alpha,z}$ and cyclical permutation under the trace.
\textcolor{red}{Details on the calculation can be found in the Appendix} The resulting differential equation is
\begin{align}
  \dot Q(\alpha,z) &= \Bigg[i\partial_z \left(-\omega_0\cos\alpha z + \frac{g}{\sqrt{j}}\left( (\cos\delta\alpha^* + \sin\delta\alpha)-(\cos\delta\alpha + \sin\delta\alpha^*)z^2\right) + \frac{\omega_0\sin\alpha}{2} (1-z^2)\right)\nonumber \\
  &~~~~~+ i\partial_\alpha \left( \omega \alpha + 2 \frac{g}{\sqrt{j}} \frac{\cos\delta z^* + \sin\delta z}{1+|z|^2} (j+1) \right)\nonumber\\
  &~~~~~+ i \partial_z \partial_\alpha \frac{g}{\sqrt{j}} \left(\cos\delta - \sin\delta z^2\right)\Bigg] Q (\alpha,z) + c.c.~~.
\end{align}
While the differential equation itself is not directly used to calculate the time evolution, we can identify drifting and diffusive terms, that influence the Husimi projection in different ways.
The drifting terms, the terms with single derivations, move the (initial) gaussian package in the complex $\alpha-z$ plain, while the diffusive term, with two derivations stretches the package and creates the quantum mechanical uncertainty.